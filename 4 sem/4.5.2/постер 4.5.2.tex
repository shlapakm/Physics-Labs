
\documentclass[a0paper,portrait]{baposter}
\usepackage[russian]{babel}
\usepackage{relsize}		% For \smaller
\usepackage{url}
\usepackage[russian]{babel}% For \url
% \usepackage{epstopdf}	% Included EPS files automatically converted to PDF to include with pdflatex

%%% Global Settings %%%%%%%%%%%%%%%%%%%%%%%%%%%%%%%%%%%%%%%%%%%%%%%%%%%%%%%%%%%

\graphicspath{{pix/}}	% Root directory of the pictures 
% \tracingstats=2			% Enabled LaTeX logging with conditionals

%%% Color Definitions %%%%%%%%%%%%%%%%%%%%%%%%%%%%%%%%%%%%%%%%%%%%%%%%%%%%%%%%%

\definecolor{bordercol}{RGB}{40,40,40}
\definecolor{headercol1}{RGB}{186,215,230}
\definecolor{headercol2}{RGB}{80,80,80}
\definecolor{headerfontcol}{RGB}{0,0,0}
\definecolor{boxcolor}{RGB}{186,215,230}

%%%%%%%%%%%%%%%%%%%%%%%%%%%%%%%%%%%%%%%%%%%%%%%%%%%%%%%%%%%%%%%%%%%%%%%%%%%%%%%%
%%% Utility functions %%%%%%%%%%%%%%%%%%%%%%%%%%%%%%%%%%%%%%%%%%%%%%%%%%%%%%%%%%

%%% Save space in lists. Use this after the opening of the list %%%%%%%%%%%%%%%%
\newcommand{\compresslist}{
	\setlength{\itemsep}{1pt}
	\setlength{\parskip}{0pt}
	\setlength{\parsep}{0pt}
}

%%%%%%%%%%%%%%%%%%%%%%%%%%%%%%%%%%%%%%%%%%%%%%%%%%%%%%%%%%%%%%%%%%%%%%%%%%%%%%%
%%% Document Start %%%%%%%%%%%%%%%%%%%%%%%%%%%%%%%%%%%%%%%%%%%%%%%%%%%%%%%%%%%%
%%%%%%%%%%%%%%%%%%%%%%%%%%%%%%%%%%%%%%%%%%%%%%%%%%%%%%%%%%%%%%%%%%%%%%%%%%%%%%%

\begin{document}
\typeout{Poster rendering started}

%%% Setting Background Image %%%%%%%%%%%%%%%%%%%%%%%%%%%%%%%%%%%%%%%%%%%%%%%%%%
\background{
	\begin{tikzpicture}[remember picture,overlay]%
	\draw (current page.north west)+(-2em,2em) node[anchor=north west]
	{\includegraphics[height=1.1\textheight]{background}};
	\end{tikzpicture}
}

%%% General Poster Settings %%%%%%%%%%%%%%%%%%%%%%%%%%%%%%%%%%%%%%%%%%%%%%%%%%%
%%%%%% Eye Catcher, Title, Authors and University Images %%%%%%%%%%%%%%%%%%%%%%
\begin{poster}{
	grid=false,
	% Option is left on true though the eyecatcher is not used. The reason is
	% that we have a bit nicer looking title and author formatting in the headercol
	% this way
	%eyecatcher=false, 
	borderColor=bordercol,
	headerColorOne=headercol1,
	headerColorTwo=headercol2,
	headerFontColor=headerfontcol,
	% Only simple background color used, no shading, so boxColorTwo isn't necessary
	boxColorOne=boxcolor,
	headershape=roundedright,
	headerfont=\Large\sf\bf,
	textborder=rectangle,
	background=user,
	headerborder=open,
  boxshade=plain
}
%%% Eye Cacther %%%%%%%%%%%%%%%%%%%%%%%%%%%%%%%%%%%%%%%%%%%%%%%%%%%%%%%%%%%%%%%
{
	Eye Catcher, empty if option eyecatcher=false - unused
}
%%% Title %%%%%%%%%%%%%%%%%%%%%%%%%%%%%%%%%%%%%%%%%%%%%%%%%%%%%%%%%%%%%%%%%%%%%
{\sf\bf
	Двойное лучепреломление
}
%%% Authors %%%%%%%%%%%%%%%%%%%%%%%%%%%%%%%%%%%%%%%%%%%%%%%%%%%%%%%%%%%%%%%%%%%
{
	\vspace{1em} Царук В В\\
}
%%% Logo %%%%%%%%%%%%%%%%%%%%%%%%%%%%%%%%%%%%%%%%%%%%%%%%%%%%%%%%%%%%%%%%%%%%%%
{
% The logos are compressed a bit into a simple box to make them smaller on the result
% (Wasn't able to find any bigger of them.)
\setlength\fboxsep{0pt}
\setlength\fboxrule{0.5pt}
	\fbox{
		\begin{minipage}{14em}
		\includegraphics[width=10em,height=4em]{LOGO2.jpg}
			\includegraphics[width=4em,height=4em]{LOGO5.png}
		\end{minipage}
	}
}

\headerbox{Цель работы}{name=1,column=0,row=0}{
Изучение зависимости показателя преломления необыкновенной волны от направления в двоякопреломляющем кристалле; определение главных показателей преломления $n_0$ --- обыкновенной и $n_e$ --- необыкновенной волны в кристалле наблюдение эффекта полного внутреннего отражения.
}
\headerbox{Приборы и материалы}{name=definitions,column=0,below=1}{
    В работе используються:Гелий-неоновый лазер, вращающийся столик с неподвижным лимбом, призма из исландского шпата, поляроид.
}

\headerbox{Эксп. установка и теория}{name=models,column=0,below=definitions}{
\begin{center}
\includegraphics[width=1\textwidth]{scheampaint.png}
\end{center}
Схема эксперементальной установки довольно проста поэтому кроме рисунка выше для понимания не надо,сложность заключаеться в теории которая очень объемна,поэтому приведем главную выдержку из нее .
Рассмотрим,как определить показатели преломления для обыкновенной и необыкновенной волны. В работе исследуется призма из исландского шпата.
в плоскости, параллельной верхней грани призмы, причем она параллельна входной грани призмы (длинному катету). При этом в обыкновенной волне вектор $\vec D_o$ пендикулярен верхней грани призмы, а в необыкновенной волне вектор $\vec D_e$  параллелен верхней грани.
Волну, падающую на входную грань призмы, можно представить в виде суммы двух ортогональных линейно поляризованных волн. Преломление этих двух волн на грани призмы можно рассматривать независимо. Волна, в которой вектор $\vec D$ направлен вертикально, внутри кристалла будет распространяться как обыкновенная. Для этой волны выполняется закон Снеллиуса, а показатель преломления призмы для нее равен $n_o$. Волна, в которой вектор $\vec D$ направлен горизонтально, в кристалле будет распространяться как необыкновенная. Для этой волны также будет выполняться закон Снеллиуса, но с тем отличием, что показатель преломления призмы для нее будет зависеть от угла между осью кристалла и волновой нормалью.
\begin{center}
\includegraphics[width=0.8\textwidth]{prismpaint.png}
\end{center}
}


\headerbox{Эксп. установка и теория}{name=density,span=2,column=1}
{
Значение показателя преломления и угол, под которым преломилась волна в призме, можно найти, измерив угол падения на входную грань призмы $\phi_1$ и угол $\phi_2$ на выходе призмы . Запишем закон Снеллиуса для одной из волн применительно к первой и второй граням призмы:
\[
	\sin \phi = n \sin \beta_1;
\]
\[
	\sin \phi_2 = n \sin \beta_2 = n \sin (A - \beta_1).
\]
При этом мы выразили угол падения на вторую грань призмы $\beta_2$ через угол преломления на первой грани призмы $\beta_1$ и угол при вершине призмы $A$. Как видно из рисунка, эти углы связаны простым соотношением $A = \beta_1 + \beta_2$. Учитывая, что угол преломления $\beta_1$ связан с углом $\theta$ между осью кристалла и волновой нормалью $\vec N$ соотношением $\theta + \beta_1 = \pi / 2$, находим $n$ и $\theta$:
\begin{equation}
n = \frac{1}{\sin A} \sqrt{\sin^2 \varphi_1 + \sin^2 \varphi_2 + 2 \sin \varphi_1 \sin \varphi_2 \cos A};
\end{equation}
\[
	\cos \theta = \frac{\sin \varphi_1}{n}.
\]
Для обыкновенной волны $n$ не будет зависеть от угла $\theta$, а для необыкновенной волны зависимость $n$ от $\theta$ должна описываться выражением (7).

Показатель преломления призмы из изотропного материала удобно находить по углу нименьшего отклонения луча от первоначального направления. Угол отклонения луча призмой (см рисунок) минимален для симметричного хода лучей, то есть когда $\varphi_1 = \varphi_2$. Тогда показатель преломления можно рассчитать по формуле
\begin{equation}
n=\frac{\sin \left(\frac{\psi_{m}+A}{2}\right)}{\sin \left(\frac{A}{2}\right)},
\end{equation}
где $\psi_m$ --- угол наименьшего отклонения.

Если призма неизотропна, то этой формулой, строго говоря, можно воспользоваться только для обыкновенной волны, которая, как это было показано ранее, распространяется так же, как и в изотропной среде. Но если учесть, что угол при вершине призмы мал, и при угле наименьшего отклонения преломлённый луч в призме распространяется под углом к оси кристалла близким к $\pi / 2$, то в качестве оценки формулу (2) можно использовать для определения $n_e$.
}

\headerbox{Данные и их обработка}{name=degreeDistribution,span=2,column=1,below=density}{
Данные можно найти по qr коду (ссылка на гугл таблицу).
\begin{center}
\includegraphics[width=1\textwidth]{graphprism.png}
\end{center}
Расчёт показателей преломления на основании трёх экспериментов:
Расчёт по обыкновенной и необыкновенной волне:
\[
	\text{При} \ A = 37^o \quad n_o = 1.65 \pm 0.01; \quad n_e = 1.49 \pm 0.03;
\] 
Расчёт по углу наименьшего отклонения:
\[
\varphi_o^m = 26.1 \pm 1; \quad \psi_e^m = 20 \pm 1.
\]
\[
	\text{При}\ A = 37^o \quad n_o = 1.64 \pm 0.003; \quad n_e = 1.50 \pm 0.04;
\]
Расчёт по углу поного внутреннего отражения:
\[
	\phi_o = (1 \pm 3)^o; \quad \phi_e = (-6.5 \pm 3)^o;
\]
\[
	n_o = 1.69 \pm 0.03; \quad n_e = 1.52 \pm 0.03.
\]
Вывод: Значения совпадают с теоретическими с учетом погрешности(ценой 3 раз перемеренных измерений)

}

\end{poster}
  \end{document}
